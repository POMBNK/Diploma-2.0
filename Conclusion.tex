\newpage
\begin{center}
	\textbf{ЗАКЛЮЧЕНИЕ}
\end{center}
%\refstepcounter{chapter}
\addcontentsline{toc}{chapter}{ЗАКЛЮЧЕНИЕ}

В рамках работы разработан прототип шагающего робота для образовательных и исследовательских целей. Были составлены чертежи и 3D модели всех составляющих робота, подобраны электрические и логические компоненты, рассчитан худший случай статической нагрузи, исходя из которого была выбрана модель сервоприводов. Исследовано движение стопы шагающего робота, описан процесс решения обратной задачи кинематики для свободного и опорного случая стопы в проблеме о положениях. Произведен синтез походки робота, решена проблема обратной связи реальных углов двигателей с помощью блока гироскопов-акселерометров. Разработано программное обеспечение на языке $Python$ 3 для управления роботом в любой доступной конфигурации, которая ограничена заранее найденной рабочей областью. В кодовой базе описаны протоколы снятия данных и последующего их преобразования, в качестве результата приведены графики реальных и идеальных значений углов поворота двигателей.

Все поставленные цели работы достигнуты, а результат проекта представлен в качестве рабочего прототипа с дистанционным управлением. 