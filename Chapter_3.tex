\newpage
\begin{center}
	\textbf{\large ГЛАВА 4 \\ МОДЕЛИРОВАНИЕ И СБОРКА}
\end{center}
\refstepcounter{chapter}


% \section*{}
\addcontentsline{toc}{chapter}{ГЛАВА 4}
%\section{Общие требования к физической модели четырехногого робота}\label{C4_1}

%Все четвероногие животные при движении сохраняют равновесие почти исключительно за счет динамической устойчивости. Изначально план чередования опорных и свободных положений ног заключался в переключении фаз по принципу ноги 1-4 опорная фаза, ноги 2-3 свободная, как показано на рисунке \ref{cycle_wanted}. Зелеными стрелками обозначены фазы свободного положения, а красными опорного. Однако, такой подход не совсем точен, хоть и вполне логичен.
%\begin{figure}[h!]
%	\begin{center}
%		\includegraphics[width=0.3\textwidth]{cycle_wanted}
%		\caption{Смена опорных и свободных фаз.}
%		\label{cycle_wanted}
%	\end{center}
%\end{figure}

%  В случае искусственных шагающих аппаратов походка должны быть определена таким образом, чтобы центр тяжести аппарата постоянно находился внутри треугольника, вершинами которого являются конечности, находящиеся на данный момент времени в опорном положении (рисунок \ref{diagrama}). На практике эта возможность была реализована д-ром Такути из Научно-исследовательского центра проблем механики\cite{Nakano}. Под его руководством был разработан шагающий аппарат с четырьмя конечностями, у которых скорость движения в фазе восстановления подобрана так, что длительность этой фазы втрое меньше длительности каждой рабочей фазы. В результате в данный момент времени лишь одна нога робота находится в воздухе, а корпус опирается на три остальные, сохраняя тем самым статическую устойчивость. 
  
%  Показанные на рисунке заштрихованные треугольники (так называемые опорные треугольники) образованы вершинами, которые соответствуют текущим точкам касания опорной поверхности какими-либо тремя из четырех ног робота. Например, в состоянии, соответствующем рисунку \ref{diagrama}, опоры касаются три ноги - А, В, С, а четвертая нога - D, будучи в фазе восстановления, находится в воздухе. Для обеспечения статической устойчивости принципиально важное значение имеет правильный выбор порядка чередования (сдвиг фаз) четырех ног в процессе движения аппарата. В момент времени, отраженный на диаграмме 1, нога В только что коснулась земли и сейчас занимает опорное положение, нога А также касается земли, но находится уже во второй половине своей рабочей фазы, а нога С - еще в первой половине собственной рабочей фазы. Нога D, находящаяся в момент наблюдения в воздухе, быстро заканчивает фазу восстановления, и к моменту времени, которому соответствует диаграмма 2, она уже опускается на землю, переходя в опорное положение. К этому моменту нога А завершила свою рабочую фазу и, оторвавшись от земли, переходит в фазу восстановления; нога В находится в первой половине, а нога С - во второй половине своих рабочих фаз. Аналогично в следующие моменты времени, которым соответствуют диаграммы 3 и 4, в фазу восстановления переходят друг за другом нога С и нога В. 
  
%  При таком порядке чередования ног в любой момент времени, когда одна из ног робота находится в фазе восстановления, центр тяжести аппарата обязательно будет лежать внутри треугольника, образованного тремя ногами, находящимися не в рабочей фазе (опирающимися на землю).
  
%\begin{figure}[h!]
%  	\begin{center}
%  		\includegraphics[width=0.3\textwidth]{diagrama}
%  		\caption{Смена опорных и свободных фаз с учетом тяжести}
%  		\label{diagrama}
%  	\end{center}
%\end{figure}

%\newpage

\section{Проектирование ног}\label{C4_2}

В разделе \ref{C1_2} были представлены особенности конструкции и основные проблемы, которые необходимо решить перед производством деталей. 

Для минимизации проблем, связанных с соосностью отверстий, отношениями и масштабами между конструкциями перед этапом печати, были использовали трехмерные твердотельные чертежи при проектировании ног в формате 3D (рисунок \ref{leg_3D}). В качестве материала для изготовления деталей ног был выбран пластик, который используется при печати с помощью 3D-принтера, поскольку уникальная конструкция ног не позволяет использовать доступные металлические изделия для сборки. Кроме того, использование пластика значительно снижает вес конструкции, что позволяет использовать менее мощные двигатели для создания прототипа. Напечатанные и собранные конструкции ног отображены на рисунках \ref{leg2}, \ref{leg3}. Характерные размеры сведены в таблице \ref{charac_table}.

\begin{table}[h]
	\begin{center}
		\caption{Характерные размеры робота}
		\label{charac_table}
		\begin{tabular}{| l | l |}
			\hline
			Длина первого звена & 100 мм\\ \hline
			Длина второго звена & 100 мм \\ \hline
			Длина лонжерона & 210 мм \\ \hline
			Длина корпуса & 210 мм \\ \hline
			Ширина корпуса & 160 мм \\ \hline
		\end{tabular}
	\end{center}
\end{table}

\newpage  
\begin{figure}[h!]
	\begin{center}
		\includegraphics[width=0.35\textwidth]{leg_3D}
		\caption{Твердотельный чертеж ноги робота}
		\label{leg_3D}
	\end{center}
\end{figure}

%\newpage
Сервоприводы, как в сочленении ног, так и в месте соединения с корпусом, закреплены к ползуну, который в свою очередь подпирается пружиной. Ползун обладает ``Т'' образной формой (рисунок \ref{polzun}), которая поддерживает его в плоскости ноги и не дает покинуть область конструкции. 

Пружина служит в качестве демпфера, который гасит ударные воздействия на вал двигателя при движении робота, а также оказывает некоторое полезное сопротивление: при резком изменении угла поворота двигателя, не дает сервоприводу развить максимальную скорость, что не раз спасало прототип при натурном тестировании походки.

\begin{figure}[h!]
	\begin{center}
		\includegraphics[width=0.8\textwidth]{polzun}
		\caption{``T'' - образный ползун}
		\label{polzun}
	\end{center}
\end{figure}
%\newpage

\begin{figure}[h!]
	\begin{center}
		\includegraphics[width=0.4\textwidth]{leg2}
		\caption{Напечатанная нога в сборке}
		\label{leg2}
	\end{center}
\end{figure}

\begin{figure}[h!]
	\begin{center}
		\includegraphics[width=0.5\textwidth]{leg3}
		\caption{Робот в исходном положении до инициализации}
		\label{leg3}
	\end{center}
\end{figure}

\newpage
\section{Описание корпуса робота}\label{C4_3}

Корпус для робота представляет из себя поперечные пластины (рисунок \ref{plastina}), соединенные с лонжеронами. Данные пластины несут функцию размещения электронных компонентов и создания жесткости конструкции. 

\begin{figure}[h!]
	\begin{center}
		\includegraphics[width=0.8\textwidth]{plastina}
		\caption{Твердотельный чертеж пластины корпуса}
		\label{plastina}
	\end{center}
\end{figure}

В качестве детали к которой крепятся ноги робота, были выбраны лонжероны, также изготовленные из пластика (рисунок \ref{longeron}). Данное решение было принято из-за легкости материала и достаточной жесткости, необходимой для удержания закрепленных деталей. 

\begin{figure}[h!]
	\begin{center}
		\includegraphics[width=0.8\textwidth]{longeron}
		\caption{Твердотельный чертеж лонжерона корпуса}
		\label{longeron}
	\end{center}
\end{figure}
\pagebreak
В отличие от ног, к корпусу были только минимальные требования, а именно небольшой вес и удобное расположение модулей для последующего подключения(подробные чертежи размещены в Приложении Б). На верхнем уровне корпуса расположены DC/DC преобразователи и трехосевой гироскоп акселерометр MPU6050. На нижнем уровне располагается миникомпьютер OrangePi 3 LTS и ШИМ контроллер PCA9685, в то время как с обратной стороны пластины закреплен источник питания в виде Li-Po аккумулятора (более подробно о подборе комплектующих в разделе \ref{C4_4}).


\section{Выбор комплектующих}\label{C4_4}
	\subsection{Выбор сервоприводов по результатам статического и динамического анализа}\label{C4_4_1}
	
Сервопривод - это механический привод, который содержит датчик (для измерения положения, скорости, усилия и других параметров) и блок управления приводом (который может быть электронной схемой или механической системой тяг), что позволяет автоматически поддерживать необходимые параметры на датчике в соответствии с заданным внешним значением (например, положение ручки управления или численное значение от других систем). Сервопривод можно рассматривать как автоматического точного исполнителя, который получает на вход значение управляющего параметра в режиме реального времени и используя показания датчика, стремится создать и поддерживать это значение на выходе исполнительного элемента. 

В данной работе предлагается рассчитать максимальный крутящий момент при худшем случае конфигурации робота. При выборе сервопривода данный параметр особенно значим, так как все остальные характеристики вроде габаритов и мест размещения крепежей у этих типов двигателей мало отличаются. Для того чтобы рассчитать худший случай нагрузки и выбрать подходящий сервопривод, необходимо знать следующие параметры:
\begin{itemize}
	\item Общий вес робота;
	\item Длины ног.
\end{itemize}

Худшей конфигурацией является случай, когда все ноги робота вытянуты в "струнку" (рисунок \ref{badass}). В таком случае серводвигателям необходимо приложить максимальный крутящий момент для возврата в рабочую конфигурацию.

\begin{figure}[h!]
	\begin{center}
		\includegraphics[width=0.5\textwidth]{badass}
		\caption{{Худшая конфигурация для статического случая}}
		\label{badass}
	\end{center}
\end{figure}

Учитывая параметры из таблицы \ref{tablParam}, а также взвесив корпус, с заранее установленными комплектующими, вес робота составляет 1.6 кг. Будем считать, что движение ног для возвращения в исходную конфигурацию будет происходить синхронно. Характер движения - двухопорная ходьба, то есть общую нагрузку можно разделить на две ноги, что составляет 0.8 кг на одну ногу. Предлагается следующая формула для вычисления максимального момента на одну ногу в случае худшей конфигурации:
 \begin{equation}
	\begin{array}{l}
		M^{\text{крут}}_{\text{макс}} = \displaystyle\frac{m_{\text{роб}}g}{4}(l_{1}+l_{2}).
	\end{array}
	\label{max_moment}
\end{equation}

Исходя из формулы \ref{max_moment} был получен крутящий момент равный 0.8 Нм. В настоящей работе выбрана модель сервопривода MG995, которая полностью удовлетворяет требованиям. Основные характеристики данного привода сведены в таблице \ref{servoParam}.


\begin{table}[h]
	\begin{center}
		\caption{Характеристики сервопривода MG995}
		\label{servoParam}
		\begin{tabular}{| l | l |}
			\hline
			Общий вес   &    55г \\ \hline
			Размеры (ШхВхГ) & 54х47.2х20мм\\ \hline
			Крутящий момент & 8.5кгс/см (4.8В), 10кгс/см (6В) \\ \hline
			Рабочая скорость & 0.2с/60$^{\circ}$ (4.8В), 0.16с/60$^{\circ}$ (6В) \\ \hline
			Рабочее напряжение & 4.8В - 7.2В \\ \hline
			Рабочая температура & 0 C$^{\circ}$  - 55 C$^{\circ}$ \\ \hline
			Зона нечувствительности ШИМ & 5мкс \\ \hline
		\end{tabular}
	\end{center}
\end{table} 


Получив наихудший статический случай, рассмотрим задачу динамики при переносе корпуса для двухопорной походки. Целью является поиск зависимости развиваемых моментов на двигателях в колене и бедре от скорости передвижения робота. В разделе \ref{C3_2} найдены законы изменения положения углов на валу двигателей в бедре и колене ($\gamma$ и $\theta$ соответственно) в зависимости от ширины шага  $X_{\text{ш}}$ и, \space следовательно, точности изменения $dx_{\text{ш}}$. Переход от зависимости номера шага $i$ к зависимости от времени $t$ получим как:

\begin{equation}
	\begin{array}{l}
		t = \displaystyle\sum_{i=0}^{2len(X_{\text{ш}})}(idt),
	\end{array}
	\label{xi_to_time}
\end{equation}
где $dt$ = 0.1 --- шаг интегрирования, $len(X_{\text{ш}})$--- количество элементов, зависящее от точности вычисления.

Используя полученные уравнения для положений, как уже известные углы, введем обобщенные координаты $\alpha$ и $\beta$ (рисунок \ref{Lagr_coord}).

\newpage
\begin{figure}[h!]
	\begin{center}
		\includegraphics[width=0.6\textwidth]{Lagr_coord}
		\caption{{Обобщенные координаты для двухзвенной ноги}}
		\label{Lagr_coord}
	\end{center}
\end{figure}

Исходя из рисунка \ref{Lagr_coord}, выразим обобщенные координаты системы как:

\begin{equation}
	\begin{array}{l}
		\alpha(t) = \displaystyle\frac{\pi}{2} + \ae + \phi,
		\\
		\beta(t)=\gamma,
	\end{array}
	\label{ab}
\end{equation}

где вспомогательные углы $\ae$,$\psi$,$\phi$ определяются как: 

\begin{equation}
	\begin{array}{l}
		\ae(t) = \displaystyle\frac{\pi}{2}-\psi,
		\\
		\psi(t) = \pi-\gamma-\phi,
		\\
		\phi(t)=\displaystyle\frac{1}{2}(\pi-\theta),
	\end{array}
	\label{add_coord}
\end{equation}

где углы $\gamma$ и $\theta$ определены в разделе \ref{C3_2}, а переход ко времени описан в уравнении \ref{xi_to_time}.

Для решения задачи поиска моментов в зависимости от скорости робота при передвижении $M(V_{\text{роб}})$, составим уравнения движения системы по обобщенным координатам $\alpha$ и $\beta$. Решая данную систему, определим моменты, действующие на систему.
Уравнения движений составим согласно уравнениям Лагранжа второго рода\cite{Lagr}.

Запишем уравнения Лагранжа второго рода в общей форме:
\begin{equation}
	\begin{array}{l}
		\displaystyle\frac{d}{dt} \displaystyle\frac{\partial L}{\partial \dot{q}_i} - \displaystyle\frac{\partial L}{\partial q_i} = Q_i,
	\end{array}
	\label{FL}
\end{equation}
где $L$ --- функция Лагранжа, которая описывает кинетическую и потенциальную энергии системы, $q_i$ --- координаты $i$-го тела в системе координат, $\dot{q}_i$ --- скорость изменения координат, а $Q_i$ --- обобщенная сила, действующая на $i$-ое тело.

Определим функцию Лагранжа как:
\begin{equation}
	\begin{array}{l}
		L = T-\Pi,		
	\end{array}
	\label{L}
\end{equation}
где $T$ --- кинетическая энергия системы, а $\Pi$ --- потенциальная энергия системы.

Запишем данные составляющие как:
\begin{equation}
	\begin{array}{l}
		T = \displaystyle\frac{1}{2}(M_{\text{роб}}+m_{\text{ног}})V^{2}_{C}+\displaystyle\frac{1}{2}m_{\text{ног}}V^{2}_{A},
		\\
		\Pi = M_{\text{роб}}g(h_{\text{1}}+h_{\text{2}}),
	\end{array}
	\label{T_P}
\end{equation}
где $h_{\text{1}} = l_{\text{ног}}\sin{(\pi-\alpha)}$ --- высота голени в зависимости от положения стопы, а $h_{\text{2}} = l_{\text{ног}}\sin{(\gamma)}$ --- высота бедра в зависимости от положения стопы.

Тогда согласно уравнению \ref{FL} запишем обобщенные силы $Q_\alpha$ и $Q_\beta$ :
\begin{equation}
	\begin{array}{l}
		Q_\alpha = M_1-M_1{_\text{тр}},
		\\
		Q_\beta = (M_2-M_2{_\text{тр}})-(M_1-M_1{_\text{тр}}), 	
		\\
		M_i{_\text{тр}} = \dot{q}_i \mu,	
	\end{array}
	\label{Q}
\end{equation}
где $M_i{_\text{тр}}$ --- момент трения в шарнирах, $\mu$ --- коэффициент трения скольжения.

Решим систему \ref{FL} с подстановками \ref{L},\ref{T_P},\ref{Q}, используя математический пакет Wolfram Mathematica. Полный листинг программы представлен в приложении В. Результатом решения получим график зависимости моментов от скорости движения робота для рабочей конфигурации $X_{\text{ш}}$ = 40 мм, точность вычисления шага $dx_{\text{ш}}$ = 10 мм  \ref{M4010}. Таким образом, изменяя ширину шага и точность вычисления, получим максимальные значения предельной конфигурации походки  $X_{\text{ш}}$ = 100 мм, точность вычисления шага $dx_{\text{ш}}$ = 16 мм \ref{Mpred}. 

\begin{figure}[h!]
	\begin{center}
		\includegraphics[width=1.\textwidth]{M4010}
		\caption{Развиваемые моменты для рабочей конфигурации}
		\label{M4010}
	\end{center}
\end{figure}

\begin{figure}[h!]
	\begin{center}
		\includegraphics[width=1.\textwidth]{Mpred}
		\caption{Развиваемые моменты для предельной конфигурации}
		\label{Mpred}
	\end{center}
\end{figure}


Вышеприведенный анализ показал, что сервопривод \ref{servoParam} полностью покрывает необходимые моменты для перемещения корпуса робота в заданной рабочей конфигурации.
 
Одним из самых главных минусов при работе с сервоприводами является гибкость управления валом двигателя. Резкие изменения угла в серводвигателях являются следствием специфики их конструкции, а если быть точнее, то данная проблема связана с принципом работы потенциометра, который выполняет роль энкодера внутри двигателя. Таким образом принцип регулирования угла поворота состоит в подаче ШИМ сигнала посредством написанного программного обеспечения, сам сигнал преобразуется в напряжение благодаря работе потенциометра и приводит вал в заданное положение с максимальной скоростью. Из этого можно сделать вывод, что в используемых  сервоприводах не имеется обратной связи, и при использовании с тяжелыми объектами будут создаваться лишние нагрузки, которые будут приводить к потреблению тока, особенно это важно по отношению к стартовым токам. В нашем случае такое поведение потенциально приведет к уничтожению редуктора внутри двигателя (рисунок \ref{servo_scheme}), так как движимые объекты инерционные и такое управление для них неприемлемо. Решение этой проблемы подробной разобрано в главе \ref{C5_2}. 

\begin{figure}[h!]
	\begin{center}
		\includegraphics[width=0.7\textwidth]{servo_scheme}
		\caption{Схема сервопривода}
		\label{servo_scheme}
	\end{center}
\end{figure}

Еще одним последствием такого принципа работы выступает осложнение корректной сборки, так как для правильной конфигурации ноги на прототипе сервоприводам требуется калибровка. Иначе говоря, необходимо выставлять крайние положения ног и рассчитывать диапазон ШИМ сигнала и принимаемого угла, чтобы как можно точнее задавать положение ноги в будущем. Это достаточно неприятное следствие, так как каждая единица серводвигателя уникальна и требует отдельной калибровки. Методы калибровки в данной работе разбираться не будут.

\subsection{Выбор управляющей электроники}\label{C4_4_2}
	
В качестве управляющей платы миникомпьютера была выбрана OrangePi 3 LTS (рисунок \ref{orange_pi}). Данный миникомпьютер покрывает все технические требования прототипа сверх нормы, что позволяет не задумываться о методах опрашивания датчиков или скорости обработки информации. Также весомым преимуществом такого компьютера является автономность и наличие встроенной ОС Linux. Такие особенности позволяют использовать плату в привычном для разработчика режиме и не терять время на настройку программного окружения. 	
\begin{figure}[h!]
	\begin{center}
		\includegraphics[width=0.4\textwidth]{orange_pi}
		\caption{OrangePi 3 LTS}
		\label{orange_pi}
	\end{center}
\end{figure}

Так как OrangePi не поддерживает управление восемью серводвигателями одновременно, было предложено использовать ШИМ контроллер. В качестве данного контроллера был выбран модуль PCA9685 (рисунок \ref{pca9685}). PCA9685 - это высокопроизводительный драйвер сервоприводов, разработанный фирмой NXP Semiconductors. Он обладает шестнадцатью независимыми каналами широтно-импульсной модуляции (PWM), каждый из которых может управлять сервоприводом. Также PCA9685 имеет встроенную функцию автоматического обновления адресов, что позволяет подключать несколько драйверов к одной шине I2C. Такой подход позволяет назначить уникальный адрес для каждого подключения, чтобы добиться одновременного управления большим количеством сервоприводов. Драйвер также поддерживает частоту ШИМ до 1 кГц, что обеспечивает высокую точность управления сервоприводами.

\begin{figure}[h!]
	\begin{center}
		\includegraphics[width=0.5\textwidth]{pca9685}
		\caption{PCA 9685}
		\label{pca9685}
	\end{center}
\end{figure}

\subsection{Выбор инерционного датчика}\label{C4_4_3}

В ходе разработки прототипа возникла проблема с отсутствием данных о реальном угле поворота двигателя. Главной причиной послужило отсутствие обратной связи у сервоприводов, о чем уже упоминалось в разделе \ref{C4_4_1}. Было предложено реализовать обратную связь с помощью использования блока гироскопов-акселерометров MPU6050 (рисунок \ref{mpu6050}).

\begin{figure}[h!]
	\begin{center}
		\includegraphics[width=0.5\textwidth]{mpu6050}
		\caption{MPU 6050}
		\label{mpu6050}
	\end{center}
\end{figure}

Принцип работы такого модуля заключается в преобразовании аналогового напряжения, считываемого с емкостных датчиков, которые затем преобразуются в цифровой сигнал в диапазоне от 0 до 32750 значений. Эти значения составляют единицы измерения для блока гироскопов-акселерометров. MPU6050 распределяет свои единицы измерения, создавая четыре уровня чувствительности, как показано рисунке \ref{mpu_raw}. Выбранный уровень чувствительности зависит от требований к точности датчика. Например, если робот собирается выполнять высокоскоростные вращения более чем на 1000° в секунду (167 об / мин), то следует установить чувствительность гироскопа на 2000°. В таком режиме гироскопу приходится преодолевать большую площадь вращения за очень короткий промежуток времени, ему необходимо экономно распределять единицы измерения. Однако для большинства задач робот вряд ли будет вращаться так быстро, поэтому необходимый и достаточный уровень чувствительности в настоящей работе - 250°, который является настройкой по умолчанию, обеспечивая очень высокий уровень точности.

\begin{figure}[h]
	\begin{center}
		\includegraphics[width=1\textwidth]{mpu_raw}
		\caption{Принцип преобразования чувствительности в MPU6050}
		\label{mpu_raw}
	\end{center}
\end{figure}
\newpage

Еще одним преимуществом данного модуля является наличие DMP (Digital Motion Processor) --- это встроенный процессор MPU6050, который объединяет данные, поступающие с акселерометра и гироскопа. С помощью данного процессора имеется возможность легко преобразовывать выходные данные сразу в углы крена и тангажа. Такой способ получения информации с датчика намного точнее, так как в процессе снятия данных происходит внутренняя фильтрация, что убирает потребность сложной фильтрации шумов в реальном времени, а также позволяет миновать пост-обработку с помощью комплиментарного фильтра, который сильно уступает по точности DMP.



\subsection{Выбор источника питания}\label{C4_4_4}
Источник питания необходимо выбирать опираясь на определенные критерии. Опишем их в порядке значимости:
\begin{itemize}
	\item Высокая токоотдача для нормальной работы сервоприводов.
	\item Большая емкость для продолжительной работы.
	\item Малый вес, чтобы не увеличивать нагрузку на двигатели.
	\item Быстрая зарядка.
\end{itemize}

Таким свойствам удовлетворяет большинство Li-Po аккумуляторов, в настоящей работе была выбрана модель Spard Li-Po 3200mAh (рисунок \ref{lipo}).
\begin{figure}[h!]
	\begin{center}
		\includegraphics[width=0.5\textwidth]{lipo}
		\caption{Li-Po аккумулятор емкостью 3200 мАч}
		\label{lipo}
	\end{center}
\end{figure}
Этот тип аккумулятора получил широкое распространение благодаря своим высоким характеристикам производительности и компактности. Тип аккумуляторов Li-Po может хранить больше энергии, чем другие типы компактных источников питания, это означает, что он может работать дольше на одном заряде, что делает его идеальным для устройств, которые требуют высокой производительности и долгого времени работы. Однако, как и все, Li-Po аккумуляторы имеют некоторые недостатки. В частности, он может быть более чувствителен к перезарядке, чем другие типы аккумуляторов.Также при возможном коротком замыкании, подключенные элементы имеют высокий шанс на полный выход из строя из-за высокого порога токоотдачи. 

%\textbf{Результаты проектирования и сборки} 
\subsection{Результаты проектирования и сборки}\label{C4_4_5}

Финальный итог сборки робота по описанным выше твердотельным чертежам c перечисленными комплектующими представлен на  рисунках \ref{side}, \ref{back}, \ref{top}. Полная схема подключения электронных компонентов (см. Приложение Г).

\begin{figure}[h!]
	\begin{center}
		\includegraphics[width=0.9\textwidth]{side}
		\caption{Собранный прототип робота - вид сбоку}
		\label{side}
	\end{center}
\end{figure}

\begin{figure}[h!]
	\begin{center}
		\includegraphics[width=0.9\textwidth]{back}
		\caption{Собранный прототип робота - вид сзади}
		\label{back}
	\end{center}
\end{figure}
\newpage
\begin{figure}[h!]
	\begin{center}
		\includegraphics[width=0.9\textwidth]{top}
		\caption{Собранный прототип робота - вид сверху}
		\label{top}
	\end{center}
\end{figure}